\section{Diagnosis}
\label{diagnosis}

\Rp and skin thickening, often with calcinosis and
telangiectasia, are commonly presented and should be ``red flags'' for
practitioners \citep{redflags,abbasi}. Skin thickening can be measured using
the modified Rodnan skin score (mRSS). mRSS is used to score the severity of
skin thickness in 17 distinct areas of the body. Each area is ranked from
normal (0) to most severe (3). Accordingly, an mRSS greater than 0 is
indicative of scleroderma in adults \citep{eurostar}. In a study by
\citeauthor{foeldvari}, the mean mRSS for healthy children was 13.92
\citep{foeldvari}. This dissimilarity between adult and juvenile scores means
clinicians should not rely on mRSS as the only diagnostic.
\citeauthor{hanitsch} documented SSc patients presenting with low mRSS scores.

Alternative diagnostics exist. Ethnocardiography and pulsed tissue Doppler can
be used for routine assessment \citep{allanore}.  Radiographs can be used to
reveal calcinosis and resorption of the distal phalangeal tufts and
demineralizations \citep{overviewSSc}. Myocardial perfusion can be assessed by
single photon emission computed tomography \citep{allanore,mele}. Cardiac MRI
should be considered because it allows for simultaneous measuring of volumes
and ventricular function, myocardial perfusion, and assessment of possible
inflammation and fibrosis \citep{allanore}. Finally, nailfold
videocapillarscopy can be used to detect the presence of giant capillaries and
microhaemorrhages, which is used to detect scleroderma \citep{cutulo}.


