\section{Manifestations} \label{manifestations}

\SSc presents in numerous ways, however, \Rp is the initial complaint in
approximately 70 percent of patients with \ssc \cite[1212]{kelley}. \Rp
results from an exaggerated cutaneous vasospastic response to temperature or
emotional events \citep{fonseca}.

\subsection{Skin involvement}

Skin involvement is characterized by scleroderma. The fingers, hands, and face
are usually the first areas of the body that are affected. In the early stages
of the disease, patients may experience pruritus and edematous swelling
\citep{overviewSSc}. The edema may regress as the disease progresses; however,
it is possible that the ability to detect it decreases as the dermis becomes
more fibrotic \citep[1213]{kelley}. \SSc may manifest itself as sclerodactyly,
digital ulcers, pitting at the finger tips, telanglectasia, and/or as
calcinosis cutis \citep{overviewSSc}.

\subsection{Vascular dysfunction}

\Rp is defined as sequential color changes in the digits caused by a change in
temperature (especially cold) or stress. The color changes are due to arterial
vasoconstriction in the digits. \Rp is characterized by color changes of
pallor (white), acrocyanosis (blue), and reperfusion hyperemia (red). It is
classically viewed as a reversible vasospasm, but many SSc patients develop
progressive structural changes in small blood vessels that cause impaired
flow. Prolonged episodes of \Rp may result in digital ulceration or infarction
\citep{overviewSSc}.

Primary \Rp has no underlying disease, and it usually has a benign course and
is reversible. Primary \Rp is common in the general population amongst
otherwise healthy individuals. Secondary \Rp occurs during the development of
a SSc \citep{overviewSSc}. Cutolo, Pizzorni, Secchi found that almost 15
percent of \Rp patients shifted from primary to secondary RP over a mean
follow-up period of 29.4$\pm$10 months \citep{cutulo}. In patients with
\lcSSc, \Rp is typically the first symptom, sometimes preceding others by
years. However, in \dcSSc, the onset of \Rp generally coincides with the
characteristic skin manifestations \citep{overviewSSc}.

\subsection{Organ involvement}

\subsubsection{Gastrointestinal involvement}

The earliest visceral manifestations of SSc are esophageal hypomotility and
lower esophageal sphincter incompetence. Approximately 90 percent of patients
with either \dcSSc or \lcSSc have gastrointestinal involvement to some extent.
Almost half of these patients may be asymptomatic, but the GI tract from mouth
to anus may be affected. Thusly, patients may experience dysphagia and
choking, heartburn, bloating, constipation, and diarrhea \citep{overviewSSc}.

\subsubsection{Pulmonary disease}

Pulmonary involvement is seen in over 70 percent of patients with SSc.
Interstitial lung disease\footnote{Also known as fibrosing alveolitis or
pulmonary fibrosis} and pulmonary vascular disease are the primary
manifestations of pulmonary involvement. Pulmonary vascular disease leads to
pulmonary arterial hypertension \citep{overviewSSc}.

\paragraph{Interstitial lung disease}

Interstitial lung disease occurs in more than 75 percent of patients with SSc.
It is commonly preceded by alveolitis. Symptoms include breathlessness on
exertion, dyspnea at rest, and a nonproductive cough. Additionally, ``Velcro''
rales are often observed \citep{overviewSSc}.

\paragraph{Pulmonary vascular disease}

Pulmonary vascular disease occur independently of interstitial lung disease.
Pulmonary vascular disease occurs in 10 to 40 percent of SSc patients but is
more common in patients with lcSSc. It is generally a late complication of
SSc. The initial symptoms are dyspnea with exertion and diminished exercise
tolerance. Severe cases can lead to cor pulmonale and right-sided heart
failure. Additionally, a frequent cause of death is thrombosis of the
pulmonary vessels, a common complication of late-stage pulmonary arterial
hypertension \citep{overviewSSc}.

\subsubsection{Renal disease}

Sixty to 80 percent of patients with \dcSSc have pathological evidence of
kidney damage. Patients may have some degree of proteinuria and/or mild
elevation in plasma creatinine concentration. Additionally, 10 to 15 percent
of patients develop severe and life-threatening renal disease. Scleroderma
renal crisis is more frequent in patients with dcSSc than those with lcSSc. It
is defined by acute onset of renal failure and mild proteinuria with few cell
or casts. Additionally, some patients may experience the sudden onset of
moderate to marked hypertension \citep{overviewSSc}.

\subsubsection{Cardiac involvement}

Sixty to 75 percent of patients with cardiac involvement due to SSc have a
prognosed mortality rate of two to five years \citep{overviewSSc}. Cardiac
involvement of \SSc includes myocardial disease, pericardial disease, and
arrhythmias.

\paragraph{Myocardial disease} 

Patchy myocardial fibrosis is the hallmark of cardiac involvement in SSc. It
may originate from recurrent vasospasm of small vessels, similar to \Rp. The
fibrosis may be increased for patients with a history of \Rp
\citep{overviewSSc}.

\paragraph{Pericardial disease}

Pericardial involvement is observed in 70 to 80 percent of SSc patients at
autopsy. Pericardial effusions may be a marker for severe SSc activity. The
presence of a large pericardial effusion could compromise cardiac output and
result in renal hypoperfusion, ultimately culminating in scleroderma renal
crisis \citep{overviewSSc}.

\paragraph{Arrhythmias}

Conduction system diseases and arrhythmias are common among SSc patients.
These are usually a result of fibrosis and can lead to death
\citep{overviewSSc}.

\subsubsection{Musculoskeletal disease}

Edema, arthralgia, and myalgia are the earliest musculoskeletal manifestations
of \dcSSc. Arthritis is uncommon in SSc, but joint pain, immobility, and
contractions are common as a result of fibrosis around joints. Contractures of
the fingers are most common, but large joint contractures involving the
wrists, elbows, and ankles may also occur.

In \dcSSc patients, palpable and/or audible deep tendon friction rubs may be
associated with the periarticular fibrosis. The most common sites involved are
the extensor and flexor tendons of the fingers and wrist and tendons over the
elbows, knees, and ankles \citep{overviewSSc}.

