\section{Management}
\label{management}

Although no disease-modifying therapy has been proven to be effective,
complications of \ssc are treatable. A number of therapies are currently in
the early stages of planning and development.

\subsection{TGF-$\beta$}

TGF-$\beta$ is required for tissue homeostasis but may also influence disease
processes including fibrosis. Inhibiting its activity may lead to aberrant
immune activation, epithelial hyperplasia, and impaired wound healing
\citep{antiTGFbeta}.

\subsubsection{Halofuginone}

Halofuginone is a plant alkaloid currently being investigated for the
treatment of skin-thickening. It interferes with transforming growth
factor-$\beta$ (TGF-$\beta$) induced collagen production
\citep{overviewTreatment}. The Food and Drug Administration granted
halofuginone orphan drug status for the treatment of scleroderma, which will
hopefully yield viable therapies \citep{halofuginone}.

\subsubsection{Caveolin-1}

Caveolin-1 is a protein that is involved in the regulation of TGF-$\beta$.
Caveolin-1 affects the development of tissue fibrosis due to this regulation;
studies have shown that reduced caveolin-1 resulted in a marked increased
collagen gene expression in normal human lung fibroblasts. It is currently
believed that coupling active caveolin-1 fragments to cell-permeable carrier
peptides could restore caveolin function, and therefore, treat SSc.
\citep{delgaldo}.

\subsection{Developing therapies}

\subsubsection{Autologous fat transplantation}

\citeauthor{roh} determined that autologous fat transplantation is effective
for long-term correction of depressed, linear scleroderma-induced atrophic
scars on the forehead. After multiple autologous fat transplantations for
depressed atrophic scar corrections, a 51--75 percent improvement was seen on
the forehead, but the chin was poor with $<$25 percent improvement. Although
transplantations to the chin, the nose, and the infraorbital area were not
effective, corrected forehead scars would improve the welfare of an SSc
patient \citep{roh}. 

\subsubsection{Ultraviolet phototheraphy}

In a recent literature review, \citeauthor{kroft} concluded that full-spectrum
UVA and UVA-1 (340--400 nm) therapy seems effective for the treatment of
sclerotic skin diseases. UVA-1 treatment can reduce the active period of
localized scleroderma and prevent further disease progression, including
contractures. Although potential long term risks of photoaging and skin
cancers are unknown, UVA-1 may prove to be a promising treatment option for
sclerotic skin \citep{kroft}.


\subsubsection{Blocking CD40-CD154 signal}

\citeauthor{kawai} suspect that blocking the CD40-CD154 signal might be an
additional way to treat the thickening of the skin. \citeauthor{kawai} found
that interactions between mast cells and fibroblasts through the CD40-CD154
signal is utilized for fibroblast activation during the early stages of SSc
\citep{kawai}.

\subsubsection{Chemotherapy}

\citeauthor{nannini} investigated the use of cyclophosphamide treatment
(chemotherapy) on pulmonary function in patients with SSc and interstitial
lung disease, but found it to be unsuccessful. After 12 months, patients had
essentially unchanged vital capacity and diffusing capacity. Currently,
chemotherapy results does not result in a clinically significant improvement
\citep{nannini}.

\subsubsection{Stem cells}

The first open-pilot, stem cell transplantation study has shown promising
results. \citeauthor{nevskaya} implanted CD34(+) cells from peripheral blood
after mobilization by G-CSF\footnote{Granulocyte colony-stimulating factor
stimulates the bone marrow to produce and release granulocytes and stem
cells.} and bone marrow into ischemic skin ulcers in hands, while mononuclear
cells were implanted in lower extremities of the same patients. CD34(+) cells
and mononuclear cells showed swift and beneficial effects on vascular symptoms
that resulted in ulcer healing, decreased frequency and duration of \Rp
attacks, and decreased ulcers and pain. Efficacy was measured by the
restoration of endothelial function, the agumentation of microcirculatory
blood flow, and the significant increase in circulating CD133(+)VEGFR2(+)
progenitors. These results demonstrate the feasibility and short-term safety
of transplantation therapy \citep{nevskaya}. Stem cell transplantation may be
a viable method of interrupting fibrogenesis \citep{laar}.

