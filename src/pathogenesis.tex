\section{Pathogenesis}
\label{pathogenesis}

Much about the pathogenesis of systemic sclerosis is unknown. Researchers
believe that SSc must have many contributing factors because its pathogenesis
involves immunologic, inflammatory, fibrotic, and vascular processes. The
abnormalities found in these processes are relatively understood, but the
initiation of these pathways remains a mystery. A number of cytokines, Vitamin
D deficiency, viruses, nitric oxide, certain autoantibodies, and
anticentromere antibodies, and environmental agents are possible contributors
\citep{SscPath}. In addition, there are genetic factors that influence disease
expression and susceptibility. 

\subsection{Effectors}

Most research focuses on understanding the interplay of fibroblasts and
endothelial cells with the cells of the immune system. Individuals with SSc
display abnormal levels of specific cytokines, antigens, and autoantibodies.
These patients often have skin T cells with increased IL-4 production and
peripheral blood T cells expressing Th-2-associated chemokine receptors. In
animal models, high levels of Th-2 cells and IL-13 can lead to muscular
hypertrophy. Researchers suspect that the high levels Th-2 cells and IL-4
could partially explain the extensive vascular involvement of the disease
\citep{chizzolini}. 

A study conducted by \citeauthor{fineschi} investigated the creation of
chemokines that are induced by IgG antifibroblast antibodies (AFAs) in
fibroblasts. In addition, the research characterized the surface molecules and
signaling paths involved. In the fibroblasts they tested, AFA-positive IgG
preferentially induced the transcription of chemokines with proangiogenic and
profibrotic potential. It was concluded that these autoantibodies bound to the
fibroblasts are likely to be important in the pathogenesis. They contribute to
the expression of the disease by up-regulating the production of proangiogenic
and profibrotic chemokines \citep{chizzolini}.

\citeauthor{szekanecz} observed that SSc patients display unusually high
levels of the following tumor-associated antigens (TAAs) CA125, CA19-9, and
CA15-3. The researchers also discovered that the serum levels of CA15-3 and
CEA were abnormally high and might correlate with renal and joint involvement
\citep{szekanecz}.

% manual break for rs2004640 
\citeauthor{dieude} created a study to assess the connection between SSc and
IRF5 \\ rs2004640 single-nucleotide polymorphism (GT) because type I
interferon has an important role in this disease and/or complications
\citep{airo}. The presence of antinuclear antibodies and fibrosing alveolitis
were strongly associated with the homozygosity for the T allele. This led the
researchers to conclude that GT substitution is linked to SSc susceptibility
\citep{dieude}.

The heat shock protein (Hsp) 70 is considered a biomarker to monitor cell
stress in SSc patients because it is commonly present in higher levels. Hsp 70
appears to be linked to skin sclerosis, renal vascular damage, inflammation,
oxidative stress, and pulmonary fibrosis \citep{ogawa}.

\subsection{Viruses}

Viruses and specific toxins have been implicated in triggering SSc in those
who are genetically-susceptible. A study by \citeauthor{prinz} explored the
potential sclerosis). The researchers analyzed ninety morphea patients
presenting with high-titer antinuclear antibodies and B burgdorferi. They
observed that a combination of an early onset of morphea combined with the
virus and high-titer antinuclear antibodies characterize a distinct
viraus-associated disease \citep{prinz}. Cytomegalovirus, Parvovirus B19,
Retroviruses, and the EpsteinÐBarr virus have also been studied. Conclusive
evidence has been found to support the involvement of Human Parvovirus B19;
SSc patients have a significantly higher occurrence rate of parvovirus than
normal controls and even other autoimmune patients \citep{ohtsuka}.

\citeauthor{batal} led a study to determine features in scleroderma renal
crisis that can be used for prognosis. The researchers reviewed the records of
seventeen patients, focusing on the results of kidney biopsies taken during
SSc renal crisis. A number of histologic features were analyzed. It was
determined that severe glomular ischemic collapse, vascular thrombosis, and
peritubular capillary C4d deposits are all indicative of an increased failure
to regain renal function \citep{batal}.

