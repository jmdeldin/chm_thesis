\section{Subtypes of \ssc}

Scleroderma is generally when skin and adjacent tissues are affected. SSc is
when disorder is associated with internal organ involvement. SSc is
subcategorized into \lcSSc and \dcSSc. The subcategories are based on the
extent and distribution of skin involvement. LcSSC is typically associated
with limited scleroderma \citep{overviewSSc}.

\subsection{Limited cutaneous SSc (lcSSc)}

Sclerosis is restricted to the hands, but face and neck are also affected to a
lesser extent. Some patients may develop sclerosis on the distal forearm.
Patients tend to have prominent vascular manifestations including sever RP and
cutaneous telangiectasia and other manifestation consistent with limited
scleroderma \citep{overviewSSc}.

\subsection{Diffuse scleroderma}

Individuals with diffuse scleroderma are more likely to present with
sclerodactyly, general skin thickening, complaints of arthritis, or evidence
of specific internal organ involvement. These symptoms may occur
contemporaneously or within 1--2 years of development of \Rp. The small number
of patients who never develop \Rp are more often male, are at a high risk for
developing renal and myocardial involvement, and have a lower survival rate
\citep[1212]{kelley}.

\subsection{Diffuse cutaneous SSc (dcSSc)}

Patients with \dcSSc have sclerotic skin in chest, abdomen or upper arms and
shoulders. They are more likely to have or to develop internal organ damage
due to ischemic injury or fibrosis than those with lcSSc \citep{overviewSSc}.

\subsection{Undifferentiated connective tissue disease}

Early ``puffy'' edamatous scleroderma, also known as undifferentiated
connective tissue disease, presents itself in the form of painless, swelling
of fingers and hands. If a patient remains at this stage, his or her long-term
prognosis is much more favorable. Symptoms of early edamatous scleroderma
include morning stiffness, arthralgia, and median nerve compression.
Additionally, pitting edema of the fingers and dorsum of hands is easily
elicited on physical examination \citep[1213]{kelley}.

Edema is typically present in unexpected locations such as upper arms, face,
and trunk. Edema is caused by the deposition of hydrophilic glycosaminoglycan
in the dermis, local inflammation, hydrostatic effects, and microvascular
disruption. In trimmed punch-skin biopsies, the percent contribution of tissue
water to weight is approximately 70 percent in all patients regardless of
clinical classification, duration of disease, or skin thickness. It is
possible that the edema does not regress; rather, the ability to detect it
decreases as the dermis becomes more fibrotic \citep[1213]{kelley}.

\subsection{Systemic sclerosis sine scleroderma}

Patients without clinically evident skin sclerosis have \ssc sine scleroderma.
It is characterized by typical vascular and/or fibrotic features of systemic
disease including renal crisis, pulmonary hypertension, and interstitial lung
disease \citep{overviewSSc}.

