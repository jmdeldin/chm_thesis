\section{Introduction}

Systemic sclerosis (scleroderma) is a remarkable connective tissue disorder.
``Sclero-'' comes from the Greek \emph{skleros} for ``hard''. It is clinically
characterized by an accelerated rate of accumulation of extracellular matrix
in both skin and internal organs, distinct immunologic abnormalities,
significant variability between acute and chronic scleroderma, and distinct
abnormalities of vascular function and structure.

% incidence
Systemic sclerosis is found in all geographic areas and in all racial groups.
The age of onset is highest between 30--50 years of age. Scleroderma is 3--4
times more common in women than in men with Women of childbearing age
\footnote{20--35 years old} at the greatest risk \citep[1212]{kelley}.  It is
estimated that 18--20 individuals per million population per year suffer from
scleroderma. In the U.S., 75,000--100,000 individuals are afflicted by it.
However, many cases of systemic scleroderma may be unrecognized or
misdiagnosed as Raynaud's phenomenon or another connective tissue disease such
as systemic lupus erythematosus \citep[1212]{kelley}.

